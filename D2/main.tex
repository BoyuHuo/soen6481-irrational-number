%%%%%%%%%%%%%%%%%%%%%%%%%%%%%%%%%%%%%%%%%%%%%%%%%%%%%%%%%%%%%%%%%%%%%
% LaTeX Template: Project Titlepage Modified (v 0.1) by rcx
%
% Original Source: http://www.howtotex.com
% Date: February 2014
% 
% This is a title page template which be used for articles & reports.
% 
% This is the modified version of the original Latex template from
% aforementioned website.
% 
%%%%%%%%%%%%%%%%%%%%%%%%%%%%%%%%%%%%%%%%%%%%%%%%%%%%%%%%%%%%%%%%%%%%%%

\documentclass[12pt]{report}
\usepackage[a4paper]{geometry}
\usepackage[myheadings]{fullpage}
\usepackage{fancyhdr}
\usepackage{lastpage}
\usepackage{graphicx, wrapfig, subcaption, setspace, booktabs}
\usepackage[T1]{fontenc}
\usepackage[font=small, labelfont=bf]{caption}
\usepackage{fourier}
\usepackage[protrusion=true, expansion=true]{microtype}
\usepackage[english]{babel}
\usepackage{sectsty}
\usepackage{url, lipsum}
\usepackage{tgbonum}
\usepackage [colorlinks,allcolors = blue]{hyperref}
\usepackage{natbib}
\usepackage{xcolor}
\usepackage{dingbat}
\usepackage{comment}


\newcommand{\HRule}[1]{\rule{\linewidth}{#1}}
\onehalfspacing
\setcounter{tocdepth}{5}
\setcounter{secnumdepth}{5}



%-------------------------------------------------------------------------------
% HEADER & FOOTER
%-------------------------------------------------------------------------------
%\pagestyle{fancy}
%\fancyhf{}
%\setlength\headheight{15pt}
%\fancyhead[L]{Student ID: 1034511}
%\fancyhead[R]{Anglia Ruskin University}
%\fancyfoot[R]{Page \thepage\ of \pageref{LastPage}}
%-------------------------------------------------------------------------------
% TITLE PAGE
%-------------------------------------------------------------------------------

\begin{document}
{\fontfamily{cmr}\selectfont
\title{ \normalsize \textsc{}
		\\ [2.0cm]
		\HRule{0.5pt} \\
		\LARGE \textbf{\uppercase{ETERNITY: NUMBERS - PI}
		\HRule{2pt} \\ [0.5cm]
		\normalsize Auguest 1, 2019 \vspace*{5\baselineskip}}
		}

\date{	Git Repo: \href{ https://github.com/BoyuHuo/soen6481-irrational-number}{ https://github.com/BoyuHuo/soen6481-irrational-number}  
}

\author{
		Baiyu Huo 40076004 \\ 
		Concordia University \\
		Department of Computer Engineering
		}

\maketitle
\tableofcontents
\newpage

%-------------------------------------------------------------------------------
% Section title formatting
\sectionfont{\scshape}
%-------------------------------------------------------------------------------

%-------------------------------------------------------------------------------
% BODY
%-------------------------------------------------------------------------------

	
\chapter{User Story} 
Global constraint: The calculation inputs and outputs should be within the range of 15 Numbers.


\section{US-1 Clear the result }
\begin{tabular}{ |p{4cm}|p{11cm}|  }
 \hline
 \multicolumn{2}{|c|}{US-1 - Clear the result } \\
 \hline
 \textbf {Story ID}& US-1 \\
  \hline
  \textbf{Priority} & LOW \\
 \hline
  \textbf{Description}   & As an user of calculator, I want to clear the previous calculation, so that I can start a new calculation from beginning \\
 \hline
 \textbf{Acceptance Criteria}& 
  \begin{itemize}
     \item User can clear the result of a calculation and start a new one
     \item It should not only clear the result number but also the calculator (in the middle of the calculation)
\end{itemize}
 
 \\
 \hline
 \textbf{Estimate} & 1 point \\
 \hline
 \textbf{Constraints}& None \\
 \hline
 \textbf{Acceptance Test}& 
  \begin{itemize}
     \item GIVEN user does calculation WHEN he/she press the clean button, THEN the calculation and result should be gone.
\end{itemize}
 
 \\
 \hline
\end{tabular}

\section{US-2 Basic calculation}
\begin{tabular}{ |p{4cm}|p{11cm}| }
 \hline
 \multicolumn{2}{|c|}{US-2 - Basic calculation} \\
 \hline
 \textbf {Story ID}& US-2 \\
 \hline
   \textbf{Priority} & HIGH \\
 \hline
  \textbf{Description}   & As an user of calculator, I want to use the basic operands so I can do the calculation such as addition, subtraction, multiplication and division  \\
\hline
     \textbf{Acceptance Criteria}& 
     \begin{itemize}
     \item Users should able to choose the operator from + ,- ,* ,/ in his or her calculation task.
     \item Users should get correct result after choose "="
     \end{itemize}
     \\
\hline
 \textbf{Estimate} & 3 points \\
 \hline
\textbf{Constrains}& the operators cannot be appeared in a row such as 5+/*3 or 3++--5 \\
 \hline
     \textbf{Acceptance Test}& 
     GIVEN user use calculator
     \begin{itemize}
          \item  WHEN he/she input 2, 3 and select "+" as operator, THEN the result should be 5.
          \item  WHEN he/she input 5, 3 and select "-" as operator, THEN the result should be 2.
         \item  WHEN he/she input 5, 3 and select "*" as operator, THEN the result should be 15.
         \item  WHEN he/she input 15, 3 and select "/" as operator, THEN the result should be 5.
     \end{itemize}
     \\
\hline

\end{tabular}


\section{US-3 Store and recover the result or number}
\begin{tabular}{ |p{4cm}|p{11cm}|  }
 \hline
 \multicolumn{2}{|c|}{US-3 - Store and recover the result and number } \\
 \hline
 \textbf {Story ID}& US-3 \\
 \hline
 \textbf{Priority} & MEDIUM \\
 \hline
  \textbf{Description}   & As an user of calculator, I want to  store a result or number into the memory so that I can recover it when I need it \\
 \hline
\textbf{Acceptance Criteria}&
   \begin{itemize}
     \item The user should able to use the stored record as a number in the calculation
     \item The original record should be erased, when the new record is saved. 
     \end{itemize}
\\
 \hline
 \textbf{Estimate} & 2 points\\
 \hline
 \textbf{Constrains}& It can only store the result number not the operators  \\
 \hline
 \textbf{Acceptance Test}&
   \begin{itemize}
     \item GIVEN user get the result of "150" WHEN he/she select "save", THEN the result "150" should be shaved in memory.
          \item GIVEN user has saved a record "150" in last calculation WHEN he/she select "memo number", THEN 150 will be used in this calculation as a number.


     \end{itemize}
\\
 \hline
\end{tabular}




\section{US-4 Get value of PI}
\begin{tabular}{ |p{4cm}|p{11cm}|  }
 \hline
 \multicolumn{2}{|c|}{US-4 - Get the value of PI} \\
 \hline
 \textbf {Story ID}& US-4  \\
 \hline
 \textbf{Priority} & HIGH \\
 \hline
 \textbf{Description}   & As an user of calculator, I want to get a value of PI and also able to change the Precision degree according to my requirement, so that I can use PI in my calculation. \\
 \hline
 \textbf{Acceptance Criteria}& 

 \begin{itemize}
     \item  The users should able to get 3.141592 when press the pi button
     \item  The users should able to use pi in any kinds calculation as a number
     \item  The users should able to get the pi in two different algorithms
     \item The users should able to get the pi in HIGH, MEDIUM, LOW level of precision.
\end{itemize}
\\
 \hline
 \textbf{Estimate} & 5 points  \\
 \hline
 \textbf{Constrains}& the display accuracy should at least keep 5 decimal places  \\
 \hline
  \textbf{Acceptance Test}& 
  \begin{itemize}
     \item  GIVEN user does the calculation WHEN he/she input pi, then the result should be 3.14159..(based on the precision degree)
     \item  GIVEN user configures the pi WHEN he/she select HIGH degree, THEN the pi should calculated to 3.141592653.
     \item  GIVEN user configures the pi WHEN he/she select MEDIUM degree, THEN the pi should calculated to 3.1415926.
\end{itemize}
\\
 \hline
\end{tabular}

\section{US-5 Calculate the area of circle }
\begin{tabular}{ |p{4cm}|p{11cm}| }
 \hline
 \multicolumn{2}{|c|}{US-5 - Calculate the area of circle } \\
 \hline
 \textbf {Story ID}& US-5  \\
 \hline
 \textbf{Priority} & MEDIUM \\
 \hline
  \textbf{Description}   & As an user of calculator, I want to calculate the are of a circle just by input the r so that I can calculate the area of circle very fast  \\
  \hline
  \textbf{Acceptance Criteria}& 
   \begin{itemize}
     \item  Users should able to input only r to get the area of a circle. 
     \item  the result should be correct and precise within 15 digital numbers.
\end{itemize}
  \\
 \hline
 \textbf{Estimate} & 3 points   \\
 \hline
 \textbf{Constrains}& the r must larger than 0 (since it doesn't make sense in the real life if the r is smaller than 0) \\
 \hline
  \textbf{Acceptance Test}& 
   \begin{itemize}
     \item  GIVEN the user calculates the area of a circle, WHEN he/she input the r = 2, THEN the result should be  12.5663704..(based on the precision degree of pi)
\end{itemize}
  \\
 \hline
\end{tabular}


\section{US-6 Calculate the circumference of the circle}
\begin{tabular}{ |p{4cm}|p{11cm}|  }
 \hline
 \multicolumn{2}{|c|}{US-6 - Calculate the circumference of the circle } \\
 \hline
 \textbf {Story ID}& US-6 \\
 \hline
 \textbf{Priority} & MEDIUM \\
 \hline
  \textbf{Description}   & As an user of calculator, I want to calculate the are of a circumference just by input the r so that I can calculate the circumference of circle very fast  \\
  \hline
  \textbf{Acceptance Criteria}&
     \begin{itemize}
     \item  Users should able to input only r to get the circumference of a circle. 
     \item  the result should be correct and precise within 15 digital numbers.
\end{itemize}
  \\
 \hline
 \textbf{Estimate} & 3 points\\
 \hline
 \textbf{Constrains}& the r must larger than 0 (since it doesn't make sense in the real life if the r is smaller than 0)   \\
 \hline
  \textbf{Acceptance Test}& 
   \begin{itemize}
     \item  GIVEN the user calculates the circumference of a circle, WHEN he/she input the r = 3, THEN the result should be 18.8495556..(based on the precision degree of pi)
\end{itemize}
  \\
 \hline
\end{tabular}

\section{US-7 Calculate the Trigonometric functions}
\begin{tabular}{ |p{4cm}|p{11cm}| }
 \hline
 \multicolumn{2}{|c|}{US-7 - Calculate the Trigonometric functions} \\
 \hline
 \textbf {Story ID}& US-7 \\
 \hline
 \textbf{Priority} & LOW \\
 \hline
  \textbf{Description}   & As an user of calculator, I want use trigonometric functions such as sine, the cosine, and the tangent, so I can do the calculation for the trigonometric problems.
  \\
  \hline
  \textbf{Acceptance Criteria}&
     \begin{itemize}
     \item  The calculation should provides functions include sine,cosine and tangent, and should able to use the pi to show the angle.
     \item  The result should be correct and precise within 15 digital numbers.
\end{itemize}
  \\
 \hline
 \textbf{Estimate} & 8 points\\
 \hline
 \textbf{Constrains}& tan pi/2 and tan pi*3/2 does not exist. \\
 \hline
   \textbf{Acceptance Test}&
     \begin{itemize}
     \item  GIVEN the user use the sin function, WHEN he/she inputs pi , THEN the result should be 0
      \item  GIVEN the user use the cosine function, WHEN he/she inputs pi , THEN the result should be -1
     \item  GIVEN the user use the tan function, WHEN he/she inputs pi/3 , THEN the result should be 1.73205080..
\end{itemize}
  \\
 \hline
\end{tabular}

\chapter{Backward Traceability Matrix } 

\section{Traceability Table}
\begin{table}[!ht]
\centering
\addtolength{\leftskip} {-2cm}
\addtolength{\rightskip}{-2cm}

\begin{tabular}{|p{2cm}|p{3cm}|p{3cm}|p{3cm}|p{3cm}|}

\hline
& Interview & Use Case & Domain Model & Online Source  \\
\hline
US-1 & & UC - Clear the result &  DM1    & \\
\hline
US-2 & & UC - Calculate the result & DM1   & \\
\hline
US-3& && &Internet URL 1\\
\hline
US-4& Interviewee: Yanpeng Wang & UC - Use the number pi& DM1  & \\
\hline
US-5& Interviewee: Yanpeng Wang & UC - Calculate the are of circle&&\\
\hline

US-6& Interviewee: Yanpeng Wang &&&\\
\hline

US-7& Interviewee: Yanpeng Wang &&&\\
\hline

\hline

\end{tabular}
\caption{Backward Traceability Matrix}
\end{table}

     \begin{itemize}
     \item Interview (US4, US5, US6, US7): Yanpeng Wang: \href{ }{
         D1 Report Chapter - 2
     }
     \item  Internet URL 1  (US3) \citep{oodesign:memento}: \href{ https://www.oodesign.com/memento-pattern-calculator-example-java-sourcecode.html 
    }{ https://www.oodesign.com/memento-pattern-calculator-example-java-sourcecode.html} 

    \item Domain model:DM1 (US1, US2, US4):  \href{ }{
     D1 Report Chapter - 4
     }
     \item Use case (US1, US2, US4, US5):  \href{ }{
     D1 Report Chapter - 5
     }

\end{itemize}


\chapter{Implementation Instruction } 

\section{Implementation Instruction}
\begin{table}[!ht]
\centering
\addtolength{\leftskip} {-2cm}
\addtolength{\rightskip}{-2cm}

\begin{tabular}{|p{4cm}|p{6cm}|}
\hline
User Story & Implemented \\
\hline
US-1 & \checkmark \\
\hline
US-2 & \checkmark \\
\hline
US-3& \checkmark\\
\hline
US-4& \checkmark\\
\hline
US-5& \checkmark\\
\hline

US-6& \checkmark\\
\hline

US-7&  \\
\hline

\hline

\end{tabular}
\caption{Project implementation table}
\end{table}
Total User Story Implementation: 6 / 7

	For more information about project implementation, please check the README file in the Git Repo.

	Git Repo: \href{ https://github.com/BoyuHuo/soen6481-irrational-number}{ https://github.com/BoyuHuo/soen6481-irrational-number}  
	

	




%-------------------------------------------------------------------------------
% REFERENCES
%-------------------------------------------------------------------------------
\newpage
	\bibliographystyle{plain}
	\bibliography{assets/references}

}
\end{document}

%-------------------------------------------------------------------------------
% SNIPPETS
%-------------------------------------------------------------------------------

%\begin{figure}[!ht]
%	\centering
%	\includegraphics[width=0.8\textwidth]{file_name}
%	\caption{}
%	\centering
%	\label{label:file_name}
%\end{figure}

%\begin{figure}[!ht]
%	\centering
%	\includegraphics[width=0.8\textwidth]{graph}
%	\caption{Blood pressure ranges and associated level of hypertension (American Heart Association, 2013).}
%	\centering
%	\label{label:graph}
%\end{figure}

%\begin{wrapfigure}{r}{0.30\textwidth}
%	\vspace{-40pt}
%	\begin{center}
%		\includegraphics[width=0.29\textwidth]{file_name}
%	\end{center}
%	\vspace{-20pt}
%	\caption{}
%	\label{label:file_name}
%\end{wrapfigure}

%\begin{wrapfigure}{r}{0.45\textwidth}
%	\begin{center}
%		\includegraphics[width=0.29\textwidth]{manometer}
%	\end{center}
%	\caption{Aneroid sphygmomanometer with stethoscope (Medicalexpo, 2012).}
%	\label{label:manometer}
%\end{wrapfigure}

%\begin{table}[!ht]\footnotesize
%	\centering
%	\begin{tabular}{cccccc}
%	\toprule
%	\multicolumn{2}{c} {Pearson's correlation test} & \multicolumn{4}{c} {Independent t-test} \\
%	\midrule	
%	\multicolumn{2}{c} {Gender} & \multicolumn{2}{c} {Activity level} & \multicolumn{2}{c} {Gender} \\
%	\midrule
%	Males & Females & 1st level & 6th level & Males & Females \\
%	\midrule
%	\multicolumn{2}{c} {BMI vs. SP} & \multicolumn{2}{c} {Systolic pressure} & \multicolumn{2}{c} {Systolic Pressure} \\
%	\multicolumn{2}{c} {BMI vs. DP} & \multicolumn{2}{c} {Diastolic pressure} & \multicolumn{2}{c} {Diastolic pressure} \\
%	\multicolumn{2}{c} {BMI vs. MAP} & \multicolumn{2}{c} {MAP} & \multicolumn{2}{c} {MAP} \\
%	\multicolumn{2}{c} {W:H ratio vs. SP} & \multicolumn{2}{c} {BMI} & \multicolumn{2}{c} {BMI} \\
%	\multicolumn{2}{c} {W:H ratio vs. DP} & \multicolumn{2}{c} {W:H ratio} & \multicolumn{2}{c} {W:H ratio} \\
%	\multicolumn{2}{c} {W:H ratio vs. MAP} & \multicolumn{2}{c} {\% Body fat} & \multicolumn{2}{c} {\% Body fat} \\
%	\multicolumn{2}{c} {} & \multicolumn{2}{c} {Height} & \multicolumn{2}{c} {Height} \\
%	\multicolumn{2}{c} {} & \multicolumn{2}{c} {Weight} & \multicolumn{2}{c} {Weight} \\
%	\multicolumn{2}{c} {} & \multicolumn{2}{c} {Heart rate} & \multicolumn{2}{c} {Heart rate} \\
%	\bottomrule
%	\end{tabular}
%	\caption{Parameters that were analysed and related statistical test performed for current study. BMI - body mass index; SP - systolic pressure; DP - diastolic pressure; MAP - mean arterial pressure; W:H ratio - waist to hip ratio.}
%	\label{label:tests}
%\end{table}%\documentclass{article}
%\usepackage[utf8]{inputenc}

%\title{Weekly Report template}
%\author{gandhalijuvekar }
%\date{January 2019}

%\begin{document}

%\maketitle

%\section{Introduction}

%\end{document}
