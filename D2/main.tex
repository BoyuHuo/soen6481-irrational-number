%%%%%%%%%%%%%%%%%%%%%%%%%%%%%%%%%%%%%%%%%%%%%%%%%%%%%%%%%%%%%%%%%%%%%
% LaTeX Template: Project Titlepage Modified (v 0.1) by rcx
%
% Original Source: http://www.howtotex.com
% Date: February 2014
% 
% This is a title page template which be used for articles & reports.
% 
% This is the modified version of the original Latex template from
% aforementioned website.
% 
%%%%%%%%%%%%%%%%%%%%%%%%%%%%%%%%%%%%%%%%%%%%%%%%%%%%%%%%%%%%%%%%%%%%%%

\documentclass[12pt]{report}
\usepackage[a4paper]{geometry}
\usepackage[myheadings]{fullpage}
\usepackage{fancyhdr}
\usepackage{lastpage}
\usepackage{graphicx, wrapfig, subcaption, setspace, booktabs}
\usepackage[T1]{fontenc}
\usepackage[font=small, labelfont=bf]{caption}
\usepackage{fourier}
\usepackage[protrusion=true, expansion=true]{microtype}
\usepackage[english]{babel}
\usepackage{sectsty}
\usepackage{url, lipsum}
\usepackage{tgbonum}
\usepackage [colorlinks,linkcolor=blue]{hyperref}
\usepackage{xcolor}


\newcommand{\HRule}[1]{\rule{\linewidth}{#1}}
\onehalfspacing
\setcounter{tocdepth}{5}
\setcounter{secnumdepth}{5}



%-------------------------------------------------------------------------------
% HEADER & FOOTER
%-------------------------------------------------------------------------------
%\pagestyle{fancy}
%\fancyhf{}
%\setlength\headheight{15pt}
%\fancyhead[L]{Student ID: 1034511}
%\fancyhead[R]{Anglia Ruskin University}
%\fancyfoot[R]{Page \thepage\ of \pageref{LastPage}}
%-------------------------------------------------------------------------------
% TITLE PAGE
%-------------------------------------------------------------------------------

\begin{document}
	{\fontfamily{cmr}\selectfont
		\title{ \normalsize \textsc{}
			\\ [2.0cm]
			\HRule{0.5pt} \\
			\LARGE \textbf{\uppercase{ETERNITY: NUMBERS - PI}
				\HRule{2pt} \\ [0.5cm]
				\normalsize July 27, 2019 \vspace*{5\baselineskip}}
		}
		
		\date{}
		
		\author{
			Baiyu Huo 40076004 \\ 
			Concordia University \\
			Department of Computer Engineering }
		
		\maketitle
		\tableofcontents
		\newpage
		
		%-------------------------------------------------------------------------------
		% Section title formatting
		\sectionfont{\scshape}
		%-------------------------------------------------------------------------------
		
		%-------------------------------------------------------------------------------
		% BODY
		%-------------------------------------------------------------------------------
		
		
		\chapter{User Story} 
		
		\section{US-1 Basic calculation}
		
		\begin{tabular}{ |p{4cm}|p{10cm}| }
			\hline
			\multicolumn{2}{|c|}{US1 - Basic calculation} \\
			\hline
			\textbf {Story ID}& US1 \\
			\hline
			\textbf{Priority} & HIGH \\
			\hline
			\textbf{Description}   & As an user of calculator, I want to use the basic operands so I can do the calculation such as addition, subtraction, multiplication and division  \\
			\hline
			\textbf{Acceptance}& 
			\begin{itemize}
				\item User can choose the operator from + ,- ,* ,/ in his or her calculation task.
				\item The result of calculation should be correct e.g. 5 + 2 = 7
				\item The priority of / and * are higher than + and -
				\item When calculate one number divided by another, the denominator cannot be 0, if it is, the result shows infinity 
			\end{itemize}
			\\
			\hline
			\textbf{Estimate} & 4 points \\
			\hline
			\textbf{Constrains}& the operators cannot be appeared in a row such as 5+/*3 or 3++--5 \\
			\hline
		\end{tabular}
		
		
		\section{US-2 Store and recover the result or number}
		\begin{tabular}{ |p{4cm}|p{10cm}| }
			\hline
			\multicolumn{2}{|c|}{US2 - Store and recover the result and number } \\
			\hline
			\textbf {Story ID}& US2 \\
			\hline
			\textbf{Priority} & MEDIUM \\
			\hline
			\textbf{Description}   & As an user of calculator, I want to  store a result or number into the memory so that I can recover it when I need it \\
			\hline
			\textbf{Acceptance}&
			\begin{itemize}
				\item User can store a number or a result any time he or she wants
				\item The recovery can be used as a number in the calculation
				\item When the 4th memory number comes in, the 1st number will be erased 
			\end{itemize}
			\\
			\hline
			\textbf{Estimate} & 2 points\\
			\hline
			\textbf{Constrains}& It can only store the result number not the operators  \\
			\hline
		\end{tabular}
		
		
		\section{US-3 Clear the result }
		\begin{tabular}{ |p{4cm}|p{10cm}| }
			\hline
			\multicolumn{2}{|c|}{US3 - Clear the result } \\
			\hline
			\textbf {Story ID}& US3 \\
			\hline
			\textbf{Priority} & LOW \\
			\hline
			\textbf{Description}   & As an user of calculator, I want to clear the previous calculation so that I can start a new calculation from beginning \\
			\hline
			\textbf{Acceptance}& 
			\begin{itemize}
				\item User can clear the result of a calculation and start a new one
				\item It can not only clear the result number but also the calculator (in the middle of the calculation)
			\end{itemize}
			
			\\
			\hline
			\textbf{Estimate} & 1 point \\
			\hline
			\textbf{Constrains}& None \\
			\hline
		\end{tabular}
		
		\section{US-4 Get value of PI}
		\begin{tabular}{ |p{4cm}|p{10cm}| }
			\hline
			\multicolumn{2}{|c|}{US4 - Get the value of PI} \\
			\hline
			\textbf {Story ID}& US4  \\
			\hline
			\textbf{Priority} & HIGH \\
			\hline
			\textbf{Description}   & As an user of calculator, I want to get a PI, which keeps at least 6 decimal places, once I click the button PI \\
			\hline
			\textbf{Acceptance}& 
			
			\begin{itemize}
				\item  the user press the pi button the number of 3.141592 should be returned
				\item  the pi can be applied in any kinds calculation as a number
			\end{itemize}
			\\
			\hline
			\textbf{Estimate} & 2 points  \\
			\hline
			\textbf{Constrains}& the display accuracy should at least keep 6 decimal places  \\
			\hline
		\end{tabular}
		
		\section{US-5 Calculate the area of circle }
		\begin{tabular}{ |p{4cm}|p{10cm}| }
			\hline
			\multicolumn{2}{|c|}{US5 - Calculate the area of circle } \\
			\hline
			\textbf {Story ID}& US5  \\
			\hline
			\textbf{Priority} & MEDIUM \\
			\hline
			\textbf{Description}   & As an user of calculator, I want to calculate the are of a circle just by input the r so that I can calculate the area of circle very fast  \\
			\hline
			\textbf{Acceptance}& 
			\begin{itemize}
				\item  there is a formula pi*R*R in the memory, the user can input only the value of r to get the result 
				\item  the result should be correct and precise within 15 digital numbers.
			\end{itemize}
			\\
			\hline
			\textbf{Estimate} & 4 points   \\
			\hline
			\textbf{Constrains}& the r must larger than 0 (since it doesn't make sense in the real life if the r is smaller than 0) \\
			\hline
		\end{tabular}
		
		
		\section{US-6 Calculate the circumference of the circle}
		\begin{tabular}{ |p{4cm}|p{10cm}| }
			\hline
			\multicolumn{2}{|c|}{US6 - Calculate the circumference of the circle } \\
			\hline
			\textbf {Story ID}& US6 \\
			\hline
			\textbf{Priority} & MEDIUM \\
			\hline
			\textbf{Description}   & As an user of calculator, I want to calculate the are of a circumference just by input the r so that I can calculate the circumference of circle very fast  \\
			\hline
			\textbf{Acceptance}&
			\begin{itemize}
				\item  there is a formula 2*pi*R in the memory, the user can input only the value of r to get the result 
				\item  the result should be correct and precise within 15 digital numbers.
			\end{itemize}
			\\
			\hline
			\textbf{Estimate} & 4 points\\
			\hline
			\textbf{Constrains}& the r must larger than 0 (since it doesn't make sense in the real life if the r is smaller than 0)   \\
			\hline
		\end{tabular}
		
		\section{US-7 Calculate the Trigonometric functions}
		\begin{tabular}{ |p{4cm}|p{10cm}| }
			\hline
			\multicolumn{2}{|c|}{US7 - Calculate the Trigonometric functions} \\
			\hline
			\textbf {Story ID}& US7 \\
			\hline
			\textbf{Priority} & LOW \\
			\hline
			\textbf{Description}   & As an user of calculator, I want use trigonometric functions such as sine, the cosine, and the tangent, so I can do the calculation for the trigonometric problems.
			\\
			\hline
			\textbf{Acceptance}&
			\begin{itemize}
				\item  the calculation should provides functions include sine,cosine and tangent, and can use the pi to show the angle.
				\item  the result should be correct and precise within 15 digital numbers.
			\end{itemize}
			\\
			\hline
			\textbf{Estimate} & 8 points\\
			\hline
			\textbf{Constrains}& the r must larger than 0 (since it doesn't make sense in the real life if the r is smaller than 0)   \\
			\hline
		\end{tabular}
		
		\chapter{Backward Traceability Matrix } 
		
		\section{Traceability Table}
		\begin{table}[!ht]
			\centering
			\addtolength{\leftskip} {-2cm}
			\addtolength{\rightskip}{-2cm}
			
			\begin{tabular}{|p{2cm}|p{4cm}|p{4cm}|p{4cm}|}
				
				\hline
				& Source 1& Source 2 & Source 3 \\
				\hline
				US-1 & Prototype in real life & Project description& \\
				\hline
				US-2 & Prototype in real life&Internet&\\
				\hline
				US-3& Prototype in real life&&\\
				\hline
				US-4& Interview: Yanpeng Wang&project description& \\
				\hline
				US-5& Interview: Yanpeng Wang &&\\
				\hline
				
				US-6& Interview: Yanpeng Wang &&\\
				\hline
				
				US-7& Interview: Yanpeng Wang &&\\
				\hline
				
				\hline
				
			\end{tabular}
			\caption{Backward Traceability Matrix}
		\end{table}
		
		\section{Traceability Source}
		
		\begin{itemize}
			\item  US-2 Internet: \href{ https://www.oodesign.com/memento-pattern-calculator-example-java-sourcecode.html}{ https://www.oodesign.com/memento-pattern-calculator-example-java-sourcecode.html} 
			
			\item Project Description:  \href{http://users.encs.concordia.ca/~kamthan/courses/soen-6481/project\_description.pdf}{
				http://users.encs.concordia.ca/~kamthan/courses/soen-6481/ project\_description.pdf
			}
			\item Interview: Yanpeng Wang: \href{ }{
				D1 Report Chapter - 2
			}
			
		\end{itemize}
		
		
		
		
		
		%-------------------------------------------------------------------------------
		% REFERENCES
		%-------------------------------------------------------------------------------
		
		
	}
\end{document}

%-------------------------------------------------------------------------------
% SNIPPETS
%-------------------------------------------------------------------------------

%\begin{figure}[!ht]
%	\centering
%	\includegraphics[width=0.8\textwidth]{file_name}
%	\caption{}
%	\centering
%	\label{label:file_name}
%\end{figure}

%\begin{figure}[!ht]
%	\centering
%	\includegraphics[width=0.8\textwidth]{graph}
%	\caption{Blood pressure ranges and associated level of hypertension (American Heart Association, 2013).}
%	\centering
%	\label{label:graph}
%\end{figure}

%\begin{wrapfigure}{r}{0.30\textwidth}
%	\vspace{-40pt}
%	\begin{center}
%		\includegraphics[width=0.29\textwidth]{file_name}
%	\end{center}
%	\vspace{-20pt}
%	\caption{}
%	\label{label:file_name}
%\end{wrapfigure}

%\begin{wrapfigure}{r}{0.45\textwidth}
%	\begin{center}
%		\includegraphics[width=0.29\textwidth]{manometer}
%	\end{center}
%	\caption{Aneroid sphygmomanometer with stethoscope (Medicalexpo, 2012).}
%	\label{label:manometer}
%\end{wrapfigure}

%\begin{table}[!ht]\footnotesize
%	\centering
%	\begin{tabular}{cccccc}
%	\toprule
%	\multicolumn{2}{c} {Pearson's correlation test} & \multicolumn{4}{c} {Independent t-test} \\
%	\midrule	
%	\multicolumn{2}{c} {Gender} & \multicolumn{2}{c} {Activity level} & \multicolumn{2}{c} {Gender} \\
%	\midrule
%	Males & Females & 1st level & 6th level & Males & Females \\
%	\midrule
%	\multicolumn{2}{c} {BMI vs. SP} & \multicolumn{2}{c} {Systolic pressure} & \multicolumn{2}{c} {Systolic Pressure} \\
%	\multicolumn{2}{c} {BMI vs. DP} & \multicolumn{2}{c} {Diastolic pressure} & \multicolumn{2}{c} {Diastolic pressure} \\
%	\multicolumn{2}{c} {BMI vs. MAP} & \multicolumn{2}{c} {MAP} & \multicolumn{2}{c} {MAP} \\
%	\multicolumn{2}{c} {W:H ratio vs. SP} & \multicolumn{2}{c} {BMI} & \multicolumn{2}{c} {BMI} \\
%	\multicolumn{2}{c} {W:H ratio vs. DP} & \multicolumn{2}{c} {W:H ratio} & \multicolumn{2}{c} {W:H ratio} \\
%	\multicolumn{2}{c} {W:H ratio vs. MAP} & \multicolumn{2}{c} {\% Body fat} & \multicolumn{2}{c} {\% Body fat} \\
%	\multicolumn{2}{c} {} & \multicolumn{2}{c} {Height} & \multicolumn{2}{c} {Height} \\
%	\multicolumn{2}{c} {} & \multicolumn{2}{c} {Weight} & \multicolumn{2}{c} {Weight} \\
%	\multicolumn{2}{c} {} & \multicolumn{2}{c} {Heart rate} & \multicolumn{2}{c} {Heart rate} \\
%	\bottomrule
%	\end{tabular}
%	\caption{Parameters that were analysed and related statistical test performed for current study. BMI - body mass index; SP - systolic pressure; DP - diastolic pressure; MAP - mean arterial pressure; W:H ratio - waist to hip ratio.}
%	\label{label:tests}
%\end{table}%\documentclass{article}
%\usepackage[utf8]{inputenc}

%\title{Weekly Report template}
%\author{gandhalijuvekar }
%\date{January 2019}

%\begin{document}

%\maketitle

%\section{Introduction}

%\end{document}
